\documentclass[11pt]{beamer}
\usetheme{EastLansing}
\usecolortheme{default}
\usepackage[utf8]{inputenc}
\usepackage{amsmath, amssymb, amsfonts, amsthm, mathtools}
\usepackage{tikz,tikz-cd}
\usepackage{graphicx}
\graphicspath{ {./images/} }
\author[K. Sreeman Reddy]{\href{http://iamsreeman.github.io/}{\textbf{Kasi Reddy Sreeman Reddy}}\linebreak\text{Undergraduate student}\linebreak\textit{Project supervised by {\href{https://www.phy.iitb.ac.in/en/employee-profile/vikram-rentala}{\normalfont Prof Vikram Rentala}}}}
\title{Category theory
applications in physics}
%\setbeamercovered{transparent} 
%\setbeamertemplate{navigation symbols}{} 
\logo{\includegraphics[width=.1\textwidth]{logo}} 
\institute[]{IIT Bombay} 
\date{Oct 2020} 
%\subject{}
\AtBeginSection[]
{
  \begin{frame}
    \frametitle{Table of Contents}
    \tableofcontents[currentsection]
  \end{frame}
}
\BeforeBeginEnvironment{definition}{
    \setbeamercolor{block title}{use=example text,fg=white,bg=example text.fg!75!black}
    \setbeamercolor{block body}{parent=normal text,use=block title example,bg=block title example.bg!10!bg}
}
\AfterEndEnvironment{definition}{
        \setbeamercolor{block title}{use=structure,fg=white,bg=structure.fg!75!black}
        \setbeamercolor{block body}{parent=normal text,use=block title,bg=block title.bg!10!bg}
}
\BeforeBeginEnvironment{theorem}{
    \setbeamercolor{block title}{use=example text,fg=white,bg=example text.fg!75!black}
    \setbeamercolor{block body}{parent=normal text,use=block title example,bg=block title example.bg!10!bg}
}
\AfterEndEnvironment{theorem}{
        \setbeamercolor{block title}{use=structure,fg=white,bg=structure.fg!75!black}
        \setbeamercolor{block body}{parent=normal text,use=block title,bg=block title.bg!10!bg}
}
%---------------------------------------------------------
\begin{document}

\begin{frame}
\titlepage
\end{frame}
\begin{frame}
\frametitle{Table of Contents}
\tableofcontents
\end{frame}
%---------------------------------------------------------


\section{Category theory}
\begin{frame}

\frametitle{What is Category theory?What are it's uses?}
Informally, category theory is a general theory of functions. Samuel Eilenberg and Saunders Mac Lane formulated the concepts of categories from 1942–45 in their study of algebraic topology, with the goal of understanding the processes that preserve mathematical structure. It may also be used as an axiomatic foundation for mathematics, as an alternative to set theory.\\
\textbf{Applications:} As we shall see a certain type of categories called \textbf{monoidal categories} are very useful in physics. The category \textbf{FdHilb} is useful in Quantum Mechanics and QFT and the category \textbf{nCob} is useful in General Relativity.
\end{frame}

\begin{frame}
\frametitle{Introduction}
\begin{definition}
\fontsize{10}{12}\selectfont
A category \textbf{C} consists of:
\begin{itemize}
    \item Objects(or nodes): A, B, C, . . . (the class of objects is denoted by \(|{\textbf{C}}|\) or ob(\textbf{C}))
    \item Morphisms (or arrows or maps): f, g, h, . . .(the set of morphisms from A to B is denoted by \textbf{C}(A,B) or hom$_{\textbf{C}}(A,B)$ )
    \item For each morphism f, there are given objects called \textit{domain} and \textit{codomain} and denoted by \textit{dom}(f),\textit{cod}(f). If $A=\textit{dom}(f)$, $B=\textit{cod}(f)$ then we write $f:A\rightarrow B$ or $f \in \textbf{C}(A,B)$ or \begin{tikzcd}[ampersand replacement=\&]
    A \arrow[r, "f"] \& B
    \end{tikzcd}
    \item $\forall f:A\rightarrow B$ \& $g:B\rightarrow C$ $\exists g \circ f:A\rightarrow C$ called a \textit{composite}  of f and g.
    \item For each object A, there is given an arrow $1_A : A \rightarrow A$ called the identity arrow of A.
\end{itemize}
and it satisfies 
\begin{itemize}
    \item \textbf{Associativity}: $h \circ (g \circ f) = (h \circ g) \circ f$  $\forall 
f:A\rightarrow B, g:B\rightarrow C,h:C\rightarrow D$.
    \item \textbf{Unit}: $f \circ 1_A = f = 1_B \circ f \forall f : A \rightarrow B$
\end{itemize}

\end{definition}
\end{frame}
\begin{frame}{Examples}
The set containing one object and one morphism is a trivial example. Another example is the category \textbf{Set}, containing all sets as objects and all functions between them as morphisms. Since functions satisfy associativity and unit properties. Some other examples-\\
\textbf{Rel}: which have all sets as objects and all relations, which essentially are subsets of the Cartesian product of 2 sets, as morphisms between them.\\
\textbf{Mon}: which have all so-called monoids, which essentially are groups without inverses, as
objects and all monoid morphisms as morphisms between them.\\
\textbf{Cat}: which have all categories as objects and all so-called functors(defined in next slide), which essentially are maps between categories, as morphisms between them.
\end{frame}
\begin{frame}{Functors}
\begin{definition}
A functor $F : \textbf{C} \rightarrow \textbf{D}$ between categories \textbf{C} and \textbf{D} is a mapping that of objects to objects and arrows to arrows, in such a way that
\begin{itemize}
    \item $F(f : A \rightarrow B) = F(f) : F(A) \rightarrow F(B)$  $\forall f \in$  hom(\textbf{C})
    \item $F(1_A) = 1_{F(A)}$ $\forall A\in $ \(|{\textbf{C}}|\)
    \item $F(g \circ f) = F(g) \circ F(f)$  $\forall f: A\rightarrow B\text{ and }g: B\rightarrow C \in$ hom(\textbf{C})
\end{itemize}
\end{definition}
It can be graphically represented as-\\
\begin{tikzcd}[ampersand replacement=\&]
A \arrow[rd,"g\circ f"] \arrow[r, "f"] \& B \arrow[d,"g"] \& \\
\& C 
\end{tikzcd}$\xrightarrow[]{\text{F}}$
\begin{tikzcd}[ampersand replacement=\&]
F(A) \arrow[rd,"F(g\circ f)"] \arrow[r, "F(f)"] \& F(B) \arrow[d,"F(g)"] \& \\
\& F(C) 
\end{tikzcd}
\end{frame}

%---------------------------------------------------------

\section{Categorical Axiomatization of
Physical systems}

%---------------------------------------------------------
%Highlighting text
\begin{frame}


\frametitle{Title}

\end{frame}

\section{No-cloning in Categorical Quantum
Mechanics}

\begin{frame}
\frametitle{Title}


\end{frame}


\end{document}