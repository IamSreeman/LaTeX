\documentclass[11pt]{beamer}
\usetheme{EastLansing}
\usecolortheme{default}
\usepackage[utf8]{inputenc}
\usepackage{amsmath, amssymb, amsfonts, amsthm, mathtools}
\usepackage{tikz,tikz-cd}
\usepackage{graphicx}
\graphicspath{ {./images/} }
\usepackage{biblatex}
\author[K. Sreeman Reddy]{\href{http://iamsreeman.github.io/}{\textbf{Kasi Reddy Sreeman Reddy}}\linebreak\text{2nd year physics student}\linebreak\text{\href{http://iamsreeman.github.io/MA109}{http://iamsreeman.github.io/MA109}}}
\title{MA 109 Tutorial 5}
%\setbeamercovered{transparent} 
%\setbeamertemplate{navigation symbols}{} 
\logo{\includegraphics[width=.1\textwidth]{logo}} 
\institute[]{IIT Bombay} 
\date{23-Dec-2020}
%\subject{}
\BeforeBeginEnvironment{definition}{
    \setbeamercolor{block title}{use=example text,fg=white,bg=example text.fg!75!black}
    \setbeamercolor{block body}{parent=normal text,use=block title example,bg=block title example.bg!10!bg}
}
\AfterEndEnvironment{definition}{
        \setbeamercolor{block title}{use=structure,fg=white,bg=structure.fg!75!black}
        \setbeamercolor{block body}{parent=normal text,use=block title,bg=block title.bg!10!bg}
}
\BeforeBeginEnvironment{theorem}{
    \setbeamercolor{block title}{use=example text,fg=white,bg=example text.fg!75!black}
    \setbeamercolor{block body}{parent=normal text,use=block title example,bg=block title example.bg!10!bg}
}
\AfterEndEnvironment{theorem}{
        \setbeamercolor{block title}{use=structure,fg=white,bg=structure.fg!75!black}
        \setbeamercolor{block body}{parent=normal text,use=block title,bg=block title.bg!10!bg}
}
%---------------------------------------------------------
\begin{document}
\begin{frame}
\titlepage
\end{frame}
%---------------------------------------------------------


\section{Sheet 5}
\begin{frame}
\frametitle{Q)2}
If $f:D\in\mathbb{R}^n\to\mathbb{R}$ be a function. Then the main difference between a level curve and a contour line is that level curve is a subset of $\mathbb{R}^n$ but contour line is a subset of $\mathbb{R}^{n+1}$.\\
(ii) For $c<0$ the level curve and contour lines are null set. For $c=0$ the level curve is the singleton set $\{(0,0)\}$ and contour line is the singleton set $\{(0,0,0)\}$. For $c>0$ the level curve is the circle with center $(0,0)$, radius $\sqrt{c}$ and lies in $\mathbb{R}^2$ and contour line is the circle with center $(0,0,c)$, radius $\sqrt{c}$, parallel to the $x-y$ plane and lies in $\mathbb{R}^3$.\\
(iii)Here if $c\neq 0$ then the level curve is a hyperbola in $\mathbb{R}^2$ and the contour is a hyperbola which is parallel to the $x-y$ plane in $\mathbb{R}^3$. If $c=0$ instead of parabola it will be a pair of straight lines.
\end{frame}
\begin{frame}
\frametitle{Q)4}
(i) We know that $(x_n,y_n)\to(x_0,y_0)\Leftrightarrow x_n\to x_0$ and $y_n\to y_0 $. Let the given function be $h(x,y)=f(x)+g(y)$. For any arbitrary sequence $(x_n,y_n)\to(x_0,y_0)$
\begin{align*}
\underset{n\to\infty}{\lim} h(x_n,y_n)&=\underset{x_n\to x_0}{\lim}f(x_n)\pm \underset{y_n\to y_0}{\lim}g(y_n)\\
\Rightarrow \underset{(x_n,y_n)\to(x_0,y_0)}{\lim} h(x,y)&=f(x_0)\pm g(y_0)\\
\end{align*}
(ii) Similar to above we can take any arbitrary sequence $(x_n,y_n)\to(x_0,y_0)$, and let $h(x_n,y_n)=f(x_n)g(y_n)$ and apply limit and it will also be continuous.
\end{frame}
\begin{frame}
(iii) and (iv) Observe that $max\{f(x),g(y)\}=\frac{f(x)+g(y)}{2}+\left|\frac{f(x)-g(y)}{2}\right|$ and $min\{f(x),g(y)\}=\frac{f(x)+g(y)}{2}-\left|\frac{f(x)-g(y)}{2}\right|$. For any continuous function $f(x,y)$, $|f(x,y)|$ is also continuous. Because if $f(x_0,y_0)>0$ or $<0$ then it is clearly continuous at that point. If $f(x_0,y_0)=0$ then also it is continuous at that point because for $-f(x_0,y_0)\leq |f(x_0,y_0)|\leq f(x_0,y_0)$ applying sandwich theorem we can say that it is continuous at this point. Now using this point and (i) we can say that $max\{f(x),g(y)\}=\frac{f(x)+g(y)}{2}+\left|\frac{f(x)-g(y)}{2}\right|$ and $min\{f(x),g(y)\}=\frac{f(x)+g(y)}{2}-\left|\frac{f(x)-g(y)}{2}\right|$ are continuous.
\end{frame}
\begin{frame}
\frametitle{Q)6(ii)}
\begin{align*}
f_x(0,0)&=\underset{h\to0}{\lim}\dfrac{f(h,0)-f(0,0)}{h}\\
&=\underset{h\to0}{\lim}\dfrac{\frac{sin^2(h)}{|h|}}{h}
\end{align*}
the right hand limit is 1 and left hand limit is -1. So it doesn't exist. We get the exact same limit for $f_y(0,0)$. So it also doesn't exist.
\end{frame}
\begin{frame}
\frametitle{Q)8}
We know that $(x_n,y_n)\to(x_0,y_0)\Leftrightarrow x_n\to x_0 and y_n\to y_0 $, we also have seen in previous tutorials that $\underset{x\to 0}{\lim}xsin(\frac{1}{x})=0$ using sandwich theorem. By using both we can say that it is continuous. Let $\textbf{v}=(v_x,v_y)$ be a unit vector. For $\textbf{x}=(0,0)$ and if $v_x,v_y\neq 0$
\begin{align*}
\nabla_{\textbf{v}}f(\textbf{x})&=\underset{h\to0}{\lim}\dfrac{f(\textbf{x}+h\textbf{v})-f(\textbf{x})}{h}\\
\Rightarrow \nabla_{\textbf{v}}f(0,0)&=\underset{h\to 0}{\lim}\dfrac{hv_xsin(\frac{1}{hv_x})+hv_ysin(\frac{1}{hv_y})}{h}
\end{align*}
this limit doesn't exist as we can get different values by taking different sequences. Similarly $f_x$,$f_y$ also don't exist. So, none of the partial derivatives exist.
\end{frame}
\begin{frame}
\frametitle{Q)10}
$\underset{(x,y)\to(x_0,y_0)}{\lim} f(x,y)=L$ iff $\forall \epsilon>0 \exists \delta >0$ such that
\begin{align*}
(x,y)\in D_f,0<\sqrt{(x-x_0)^2+(y-y_0)^2}<\delta\Rightarrow |f(x,y)-L|<\epsilon
\end{align*}
Here for $L=0$ $\delta=\epsilon$ works $\forall \epsilon>0$ as $|f(x,y)-L|=\sqrt{x^2+y^2}$. So it is continuous at $(0,0)$. Let $\textbf{v}=(v_x,v_y)$ be a unit vector. For $\textbf{x}=(0,0)$ and if $v_y\neq 0$
\begin{align*}
\nabla_{\textbf{v}}f(\textbf{x})&=\underset{h\to0}{\lim}\dfrac{f(\textbf{x}+h\textbf{v})-f(\textbf{x})}{h}\\
\Rightarrow \nabla_{\textbf{v}}f(0,0)&=\underset{h\to 0}{\lim}\dfrac{hv_y\sqrt{h^2(1)}}{h|hv_y|}=\underset{h\to 0}{\lim}\dfrac{hv_y|h|}{h|h||v_y|}=\dfrac{v_y}{|v_y|}
\end{align*}
If $v_y=0 \Rightarrow v_x=\pm 1\Rightarrow\nabla_{\textbf{v}}f(\textbf{x})=\frac{0-0}{h}=0$
\end{frame}
\begin{frame}
From above we get $f_x(0,0)=0,f_y(0,0)=1$ if it differentiable at $(0,0)$ then $\exists \alpha ,\beta$ such that
\begin{align*}
\underset{(h,k)\to(0,0)}{\lim} \dfrac{f(x_0+h,y_0+k)-f(x_0,y_0)-\alpha h-\beta k}{\sqrt{h^2+k^2}}=0
\end{align*}
and $\alpha,\beta$ coincides with the $x,y$ directional derivatives. Here
\begin{align*}
\underset{(h,k)\to(0,0)}{\lim} \dfrac{\frac{k}{|k|}\sqrt{h^2+k^2}-k}{\sqrt{h^2+k^2}}=0
\end{align*}
It should be 0 along all sequences converging to (0,0). Take a sequence along the line $x=y$ approaching from the 1st quadrant. The limit will \\be $\frac{\sqrt{2}-1}{\sqrt{2}-1}$, so it is not differentiable.
\end{frame}
\begin{frame}
\frametitle{Some important points}
\begin{enumerate}
  \item Differentiable at $(x_0,y_0)$.
  \item $\nabla_{\textbf{v}}f(\textbf{x})=(f_x(x_0,y_0),f_y(x_0,y_0))\cdot \textbf{v}$.
  \item The directional derivative exists for all $\textbf{v}$.
  \item $f_x$ and $f_y$ exist.
  \item $f$ is continuous $(x_0,y_0)$
\end{enumerate}
$$(i)\Rightarrow (ii)\Rightarrow (iii)\Rightarrow (iv)\nRightarrow (v)$$
$$(v)\nRightarrow (iv)\nRightarrow (iii)\nRightarrow (ii)\nRightarrow (i)$$
\end{frame}
\end{document}