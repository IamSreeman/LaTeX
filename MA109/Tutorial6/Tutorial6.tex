\documentclass[11pt]{beamer}
\usetheme{EastLansing}
\usecolortheme{default}
\usepackage[utf8]{inputenc}
\usepackage{amsmath, amssymb, amsfonts, amsthm, mathtools}
\usepackage{tikz,tikz-cd}
\usepackage{graphicx}
\graphicspath{ {./images/} }
\usepackage{biblatex}
\author[K. Sreeman Reddy]{\href{http://iamsreeman.github.io/}{\textbf{Kasi Reddy Sreeman Reddy}}\linebreak\text{2nd year physics student}\linebreak\text{\href{http://iamsreeman.github.io/MA109}{http://iamsreeman.github.io/MA109}}}
\title{MA 109 Tutorial 6}
%\setbeamercovered{transparent} 
%\setbeamertemplate{navigation symbols}{} 
\logo{\includegraphics[width=.1\textwidth]{logo}} 
\institute[]{IIT Bombay} 
\date{30-Dec-2020}
%\subject{}
\BeforeBeginEnvironment{definition}{
    \setbeamercolor{block title}{use=example text,fg=white,bg=example text.fg!75!black}
    \setbeamercolor{block body}{parent=normal text,use=block title example,bg=block title example.bg!10!bg}
}
\AfterEndEnvironment{definition}{
        \setbeamercolor{block title}{use=structure,fg=white,bg=structure.fg!75!black}
        \setbeamercolor{block body}{parent=normal text,use=block title,bg=block title.bg!10!bg}
}
\BeforeBeginEnvironment{theorem}{
    \setbeamercolor{block title}{use=example text,fg=white,bg=example text.fg!75!black}
    \setbeamercolor{block body}{parent=normal text,use=block title example,bg=block title example.bg!10!bg}
}
\AfterEndEnvironment{theorem}{
        \setbeamercolor{block title}{use=structure,fg=white,bg=structure.fg!75!black}
        \setbeamercolor{block body}{parent=normal text,use=block title,bg=block title.bg!10!bg}
}
%---------------------------------------------------------
\begin{document}
\begin{frame}
\titlepage
\end{frame}
%---------------------------------------------------------


\section{Sheet 6}
\begin{frame}
\frametitle{Q)2}
The given function $f(x,y)=x^2+sin(xy)$ is a differentiable function. We know that for a differentiable function
$$\nabla_{\textbf{v}}f(\textbf{x})=(f_x(x_0,y_0),f_y(x_0,y_0))\cdot \textbf{v}$$
for all unit vectors $\textbf{v}$. 
$$f_x(x_0,y_0)=2x_0+y_0cos(x_0y_0)\Rightarrow f_x(1,0)=2$$
$$f_y(x_0,y_0)=x_0cos(x_0y_0)\Rightarrow f_y(1,0)=1$$
without loss of generality we can take $\textbf{v}=(cos\theta,sin\theta)$.
\begin{align*}
\nabla_{\textbf{v}}f(\textbf{x})=(2,1)\cdot (cos\theta,sin\theta)&=1\\
\implies 2cos\theta+sin\theta&=1
\end{align*}
\end{frame}
\begin{frame}
By defining $\alpha=sin^{-1}(\frac{1}{\sqrt{5}})$ we get
\begin{align*}
 \sqrt{5}(\frac{2}{\sqrt{5}}cos\theta+\frac{1}{\sqrt{5}}sin\theta)&=1\\
 \implies cos(\theta-\alpha)&=\frac{1}{\sqrt{5}}\\
 \implies \theta-\alpha&=cos^{-1}(\frac{1}{\sqrt{5}}) \text{ or } 2\pi-cos^{-1}(\frac{1}{\sqrt{5}})\\
 \implies \theta-\alpha&=\frac{\pi}{2}-\alpha \text{ or } \frac{3\pi}{2}+\alpha \\
  \implies \theta&=\frac{\pi}{2} \text{ or } \frac{3\pi}{2}+2\alpha \\
  \textbf{v}&=(0,1)\text{ or }(\frac{4}{5},-\frac{3}{5})
\end{align*}
\end{frame}
\begin{frame}
\frametitle{Q)4}
$\textbf{u}=(\frac{2}{3},\frac{2}{3},\frac{1}{3})$. We can also see that $f_x(x_0,y_0,z_0)=3$,$f_y(x_0,y_0,z_0)=-5$ and $f_z(x_0,y_0,z_0)=2$
\begin{align*}
\nabla_{\textbf{u}}f(\textbf{x})&=(f_x(x_0,y_0,z_0),f_y(x_0,y_0,z_0),f_z(x_0,y_0,z_0))\cdot \textbf{u}\\
\Rightarrow\nabla_{\textbf{u}}f(2,2,1)&=(3,-5,2)\cdot (\frac{2}{3},\frac{2}{3},\frac{1}{3})\\
\Rightarrow\nabla_{\textbf{u}}f(2,2,1)&=-\frac{2}{3}
\end{align*}
\end{frame}
\begin{frame}
\frametitle{Q)5}
\begin{align*}
sin(x+y)+sin(y+z)&=1\\
\Rightarrow cos(x+y)+cos(y+z)\dfrac{\partial z}{\partial x}&=0\\
\Rightarrow \dfrac{\partial z}{\partial x}&=-\dfrac{cos(x+y)}{cos(y+z)}\text{ since $cos(y+z)\neq 0$}
\end{align*}
By partial differetiating the initial equation by $y$ we get
\begin{align*}
cos(x+y)+cos(y+z)\left(1+\dfrac{\partial z}{\partial y}\right)&=0
\end{align*}
By partial differetiating the above equation by $x$ we get
\end{frame}
\begin{frame}
\begin{align*}
-sin(x+y)-sin(y+z)\left(1+\frac{\partial z}{\partial y} \right)\frac{\partial z}{\partial x}+cos(y+z)\dfrac{\partial^2 z}{\partial x\partial y} &=0
\end{align*}
By substituting the values of $\frac{\partial z}{\partial x}$ and $\frac{\partial z}{\partial y}$ we get
$$\dfrac{\partial^2 z}{\partial x\partial y}=\dfrac{sin(x+y)}{cos(y+z)}+tan(y+z)\dfrac{cos^2(x+y)}{cos^2(y+z)}$$
\end{frame}
\begin{frame}
\frametitle{Q)8}
Here we can use second derivative test or discriminant test.\\
For the (i) part we can apply the test at all the critical points.\\
For the (ii) part the test fails at the critical point $(0,0)$ as the discriminant will be 0. But if we fix $y=0$, we can see that $f(x,0)=x^3$ and we also know that $g(x)=x^3$ is strictly increasing at $x=0$. We can conclude that $(0,0)$ is a saddle point. 
\end{frame}
\begin{frame}
\frametitle{Q)9}
If you write $g(x,y)=-f(x,y)=(4x-x^2)(cos y)$ in the given domain $3\leq4x-x^2\leq 4$ and $\frac{1}{\sqrt{2}}\leq cosy\leq 1$. For minimum of $g(x,y)$(which will be negative of maximum of $f(x,y)$) we should take both minimum values and for maximum of $g(x,y)$(which will be negative of minimum of $f(x,y)$) both maximum values.\\ 
Global maximum of f(x,y) is $-\dfrac{3}{\sqrt{2}}$\\
Global minimum of f(x,y) is $-4$\\
We can also do this with the second derivative test and observing the boundaries.
\end{frame}
\begin{frame}
\frametitle{}
All the best, not just for the MA 109 final exam (which is on 6th January) but for all other exams also. Be careful and don't do any silly mistakes.
\end{frame}
\end{document}